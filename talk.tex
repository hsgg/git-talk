\documentclass{beamer}

\usepackage[utf8]{inputenc}
\usepackage[T1]{fontenc}

\usepackage{lmodern} % get rid of fontsize warning
\usepackage{tikz}
%\usetikzlibrary{calc,chains,shapes,positioning}
\usepackage{pgfplots}
\usepackage{verbatim}
\usepackage{amsmath}
\usepackage{textpos}
\usepackage{graphicx}
\usepackage[normalem]{ulem}
%\usepackage[colorlinks=false,pdfborder={0 0 0}]{hyperref}



\title{git - the stupid content tracker}
\subtitle{git eats trees.}
\author{Henry S. G. Gebhardt}
%\institute[The Pennsylvania State University]
%{Department of Astronomy and Astrophysics}
%\date{February 26, 2013}



\newcommand{\comp}[1]{{\tt #1}}


\begin{document}

\frame{\titlepage}

\section[Outline]{}
\frame{
    \frametitle{Outline}
    \tableofcontents
}


\section{Introduction to Version Control}
\frame{
    \frametitle{Version Control? Version Control.}

    \begin{itemize}
        \item Save history of plain-text files.
        \item Keep track of changes.
        \item Merge code.
        \item Share code. (Don't be a git!)
        \item Code integrity.
    \end{itemize}

    Git, Mercurial, \sout{Bazaar}, \sout{SVN}, \sout{CVS},
    \sout{Monotone}, \sout{DARCS}, \ldots

    ``Theory of Patches''

    What is a patch?
}


\section{Theory}
\frame{
    \frametitle{Git Theory}

    \begin{center}
        \url{https://git-scm.com}
    \end{center}
    \bigskip

    History is a DAG (directed acyclic graph). \emph{Explain graph.}
    \bigskip

    Distributed, not centralized. \emph{Every clone has the full history.}
    \bigskip

    There are \emph{plumbing} commands and \emph{porcelain} commands.
}

\frame{
    \frametitle{Git doesn't know about directories\ldots whaaat?}

    Git only knows content. (blobs)
    \bigskip

    How that content is assembled. (trees)
    \bigskip

    And history. (commits)
}


\frame{
    \frametitle{blobs, trees, and commits are identified by their SHA1-sum}

    A hash is a (hopefully) unique number to identify some information, like a
    file.
    \medskip

    SHA-1 is a 160-bit number. It happens to be cryptographically secure.
    \medskip

    Blobs, trees, and commits are identified by their SHA1 sum.
    \medskip

    $\Rightarrow$ Efficient de-duplication and compression

    $\Rightarrow$ Every commit stores the entire state of the project.
}


\frame{
    \frametitle{git stores its information in a .git directory}
    {\tt
        \hspace{1cm}newrepo/ \\
        \hspace{1.5cm}    .git/ \\
        \hspace{1.5cm}    .gitignore \\
        \hspace{1.5cm}    README.md \\
        \hspace{1.5cm}    doc/ \\
        \hspace{1.5cm}    src/ \\
        \hspace{1.5cm}    test/ \\
    }
}


\frame{
    \frametitle{Per-seat Initialization}
    Initialization once per machine ({\tt \char`~/.gitconfig}):
    \begin{quote}
        \tt
        \$ git config -$\,$-global user.name "Henry VIII" \\
        \$ git config -$\,$-global user.email "h@here.com"
    \end{quote}

    \bigskip
    Set your {\tt EDITOR} variable in {\tt \char`~/.bashrc} or {\tt \char`~/.profile}.
}


\frame{
    \frametitle{Making history}
    First, add changes to staging area (the \emph{index}):
    \begin{description}[Other]
        \item[git add <file>]  \# Add your changes to the index.
        \item[git add -p] \hspace{0.6cm} \# Be selective about what to add.
    \end{description}

    \bigskip
    Then, commit:
    \begin{description}[Other]
        \item[git commit]  \# Commits your changes.
    \end{description}
}


\section{Workflows and git branches}
\frame{
    \frametitle{What to commit}
    \begin{itemize}
        \item The bare minimum to recreate the project.
        \item Plots! (They are the minimum to recompile the \LaTeX{}.)
    \end{itemize}

    \bigskip
    \bigskip
    \texttt{.gitignore} can appear anywhere in the git repository, and contains
    files to ignore, e.g.,
    \begin{description}[Other]
        \item[.gitignore]\phantom{0}\\
            *.aux \\
            *.log \\
            *.toc \\
    \end{description}

    \bigskip
    Reasons: avoid conflicts, save space.
}


\frame{
    \frametitle{Where to commit: Branches}
    \begin{description}
        \item[\emph{master} (or \emph{main}):] Main development branch.
        \item[release branches:] More stable branches that you might want to
            keep supporting by fixing bugs and cherry-picking commits from
            \emph{master}.
        \item[feature branches:]  Development branch for specific features. By
            convention, they often start with your initials, e.g.,
            \emph{hg/integration\_method\_B}. Should eventually be merged into
            \emph{master}.
    \end{description}

    %\frametitle{Trees, yum!}
    %\framesubtitle{git eats trees\ldots nom,nom}

    %Branches are cheap!

    \bigskip
    \begin{description}
        \item[git branch -a] List branches.
        %\item[git branch -d <name>] Delete a branch.
        %\item[git checkout <name>] Let's climb over to that branch.
        \item[git checkout {[}-b <newbranch>{]} <starthere>] Checkout and make a new branch.
        \item[git merge <otherbranches>...] Trees eating trees!
        \item[git rebase -i <branchname>] Clean up your history!
    \end{description}
}


\section{Clones and Remote Repositories}
\frame{
    \frametitle{Each git clone is a full repository}

    \includegraphics[width=\textwidth]{Distributed-Version-Control-System-Workflow-What-Is-Git-Edureka.png}

    \texttt{git pull} = \texttt{git fetch} + \texttt{git merge}
}


\frame{
    \frametitle{Pull Requests (PR)}
    PRs are used to coordinate work between multiple people.
}


\frame{
    \frametitle{Hosting your git repository}
    Companies:
    \begin{itemize}
        \itemindent3em
        \item[Github:] \url{github.com}
        \item[Gitlab:] \url{gitlab.com}
        \item[\ldots]
    \end{itemize}
    \bigskip
    Your own:
    \begin{itemize}
        \itemindent3em
        \item[SSH server:] Your own workstation/server. (\emph{git init -$\,$-bare})
        \item[SSH server:] \url{http://gitolite.com/} (probably overkill)
        \item[\ldots]
    \end{itemize}
    %\bigskip
    %{\tt
    %    \hspace{1cm}\$ mkdir -p \char`~/repos/newawesomeproject.git\\
    %    \hspace{1cm}\$ cd \char`~/repos/newawesomeproject.git\\
    %    \hspace{1cm}\$ git init -$\,$-bare
    %}
}



\section{Summary and cheatsheet}

\frame{
    \frametitle{git cheat sheet}

    Here's the \emph{porcelain}:
    \bigskip

    %\url{https://services.github.com/kit/downloads/github-git-cheat-sheet.pdf}
    \url{https://about.gitlab.com/images/press/git-cheat-sheet.pdf}

    \vfill

    {\tt man gittutorial}

    \vfill
    Initialization once per machine:\\
    \hspace{1cm}Create the file {\tt \char`~/.gitconfig}.

    Set your {\tt EDITOR} variable in {\tt \char`~/.bashrc} or {\tt \char`~/.profile}.
}


\appendix
\frame{
    \centering \huge \color{blue}{Appendix}
}

\frame{
    \frametitle{git help command}

    Useful commands:

    \begin{description}[Other]
        \item[git status] Where am I?
        \item[git diff] What did I just do?
        \item[git diff -$\,$-staged] What will I do?
        \item[git log] What have I done?
        \item[gitk -$\,$-all] Let's climb trees!
        \item[git describe -$\,$-always -$\,$-tags -$\,$-dirty] Who am I?
    \end{description}
}

\frame{
    \frametitle{Pushing and pulling}
    {\tt
        \hspace{1cm}\$ git push <remote> <localbranch>:<remotebranch> \\
        \hspace{1cm}\$ git push -$\,$-set-upstream \\
        \hspace{1cm}\$ git pull \\
        \hspace{1cm}\$ git remote -v \\
    }
}

\frame{
    \frametitle{Initial checkout}
    Existing repository:
    {\tt\\
        \hspace{0.3cm}\$ git clone <url>
    }
    \vfill
    New repository:
    {\tt\\
        \hspace{1cm}\$ mkdir newrepo; cd newrepo\\
        \hspace{1cm}\$ git init
    }
}

\frame{
    \frametitle{Sending and receiving patches}
    \begin{description}[Other]
        \item[git format-patch] Create a patch
        \item[git send-email] Send an entire set of patches as emails.
        \item[git am, git apply] Apply other people's patches.
    \end{description}
}

\frame{
    \frametitle{Merging branches and conflicts}
    {\tt git mergetool}
}


\frame{
    \frametitle{Play with me!}
    \begin{center}
        {\tt \$ git svn}
        \vfill
        Works by calling ``{\tt git fast-import}''.
    \end{center}
}


\frame{
    \frametitle{Ah, I did something stupid\ldots}

    Recovery might be possible by looking into {\tt .git/logs/}.
}


\frame{
    \frametitle{Other commands}

    Graphs: {\tt git log -$\,$-graph}
    \bigskip

    More graphs: {\tt gitk -$\,$-all}
    \bigskip

    Tags: {\tt git tag}
    \bigskip

    Hooks: {\tt man githooks}; {\tt cd .git/hooks/}
    \bigskip

    Submodules: {\tt git submodule}
    \bigskip

    Rewrite history: {\tt  git filter-branch}
    \bigskip

    Collect garbage: {\tt git gc}
}



\end{document}


% vim: set sw=4 sts=4 et:
