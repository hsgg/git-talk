\documentclass{beamer}

\usepackage[utf8]{inputenc}
\usepackage[T1]{fontenc}

\usepackage{lmodern} % get rid of fontsize warning
\usepackage{tikz}
%\usetikzlibrary{calc,chains,shapes,positioning}
\usepackage{pgfplots}
\usepackage{verbatim}
\usepackage{amsmath}
\usepackage{textpos}
\usepackage{graphicx}
\usepackage[normalem]{ulem}
%\usepackage[colorlinks=false,pdfborder={0 0 0}]{hyperref}



\title{git - the stupid content tracker}
\subtitle{git eats trees. version control with git is fun.}
\author{Henry S. G. Gebhardt}
%\institute[The Pennsylvania State University]
%{Department of Astronomy and Astrophysics}
%\date{February 26, 2013}



\newcommand{\comp}[1]{{\tt #1}}


\begin{document}

\frame{\titlepage}

\section[Outline]{}
\frame{
    \frametitle{Outline}
    \tableofcontents
}


\section{Introduction to Version Control}
\frame{
    \frametitle{Version Control? Version Control.}

    \begin{itemize}
        \item Save history.
        \item Keep track of changes.
        \item Merge code.
        \item Don't be a git. Share code.
        \item \ldots
    \end{itemize}

    Git, Mercurial, \sout{Bazaar}, \sout{SVN} (why bother?), \sout{CVS},
    \sout{Monotone}, \sout{DARCS}, \ldots

    ``Theory of Patches''

    What is a patch?
}


\frame{
    \frametitle{Why version control?}
    \begin{itemize}
        \item Not all VCS are worth using\ldots
        \item Keep a backup
        \item Keep track of changes
        \item Keep multiple versions.
        \item Collaborate
        \item Revert changes
        \item Blame people
    \end{itemize}
}


\section{Theory}
\frame{
    \frametitle{Git Theory}

    \begin{center}
        \url{https://git-scm.com}
    \end{center}
    \bigskip

    History is a DAG (directed acyclic graph). \emph{Explain graph.}
    \bigskip

    Distributed, not centralized. \emph{Every clone has the full history.}
    \bigskip

    There are \emph{plumbing} commands and \emph{porcelain} commands.
}

\frame{
    \frametitle{Git doesn't know about files\ldots whaaat?}

    Git only knows content. (blobs)
    \bigskip

    And how that content is assembled. (trees)
    \bigskip

    And history. (commits)
}


\frame{
    \frametitle{blobs, trees, and commits are identified by their SHA1-sum}

    A hash is a (hopefully) unique number to identify some information, like a
    file.
    \medskip

    SHA-1 is a 160-bit number. It happens to be cryptographically secure.
    \medskip

    Blobs, trees, and commits are identified by their SHA1 sum.
    \medskip

    $\Rightarrow$ Efficient de-duplication and compression
}


\frame{
    \frametitle{Terminology: blah, blah, blah,\ldots}
    WORKDIR \\
    GITDIR \\
    HEAD \\
    Index \\
    Local repository \\
    Upstream repository \\
    Stash \\
    branch \\
    master branch
}


\section{git commands}
\frame{
    \frametitle{git cheat sheet}

    Here's the \emph{porcelain}:
    \bigskip

    \url{https://services.github.com/kit/downloads/github-git-cheat-sheet.pdf}
    \bigskip

    {\tt man gittutorial}

    \vfill
    Initialization once per machine:\\
    \hspace{1cm}Create the file {\tt \char`~/.gitconfig}.

    Set your {\tt EDITOR} variable in {\tt \char`~/.bashrc}.
}


\frame{
    \frametitle{Initial checkout}
    Existing repository:
    {\tt\\
        \hspace{0.3cm}\$ git clone ssh://git@git.psu.edu/hsg113/test.git
    }
    \vfill
    New repository:
    {\tt\\
        \hspace{1cm}\$ mkdir newrepo; cd newrepo\\
        \hspace{1cm}\$ git init
    }
}


\frame{
    \frametitle{git help command}

    Useful commands:

    \begin{description}[Other]
        \item[git status] Where am I?
        \item[git diff] What did I just do?
        \item[git diff -$\,$-staged] What will I do?
        \item[git log] What have I done?
        \item[gitk -$\,$-all] Let's climb trees!
        \item[git describe -$\,$-always -$\,$-tags] Who am I?
    \end{description}
}

\frame{
    \frametitle{Commits}
    \begin{description}[Other]
        \item[git add <file>] Add your changes to the index.
        \item[git add -p] Be selective about what to add.
        \item[git commit] Commit your changes.
    \end{description}
}

\frame{
    \frametitle{Sending and receiving patches}
    \begin{description}[Other]
        \item[git format-patch] Create a patch
        \item[git send-email] Send an entire set of patches as emails.
        \item[git am, git apply] Apply other people's patches.
    \end{description}
}

\frame{
    \frametitle{Trees, yum!}
    \framesubtitle{git eats trees\ldots nom,nom}

    Branches are cheap!

    \begin{description}
        \item[git branch <name>] Let's make a new branch.
        \item[git branch -d] Never mind.
        \item[git checkout <name>] Let's climb over to that branch.
        \item[git checkout -b <newname> <starthere>] Checkout and make a new branch.
        \item[git merge <otherbranches>...] Trees eating trees!
        \item[git rebase -i <branchname>] Clean up your history!
    \end{description}
}


\frame{
    \frametitle{Pushing and pulling}
    {\tt
        \hspace{1cm}\$ git push <remote> <localbranch>:<remotebranch> \\
        \hspace{1cm}\$ git push -$\,$-set-upstream \\
        \hspace{1cm}\$ git pull \\
        \hspace{1cm}\$ git remote -v \\
    }
}


\frame{
    \frametitle{Workflow}
    {\tt git mergetool}
}


\frame{
    \frametitle{Play with me!}
    \begin{center}
        {\tt \$ git svn}
        \vfill
        Works by calling ``{\tt git fast-import}''.
    \end{center}
}


\frame{
    \frametitle{Hosting your git repository}
    Others:
    \begin{itemize}
        \itemindent3em
        \item[Github:] \url{github.com}, \url{gitorious.com}, \ldots
        \item[PSU:] \url{git.psu.edu}
    \end{itemize}
    \bigskip
    Your own:
    \begin{itemize}
        \itemindent3em
        \item[SSH server:] \url{hartmann.astro.psu.edu}, \ldots your own workstation
        \item[SSH server:] \url{http://gitolite.com/gitolite/} (probably overkill)
    \end{itemize}
    \bigskip
    {\tt
        \hspace{1cm}\$ mkdir -p \char`~/repos/newawesomeproject.git\\
        \hspace{1cm}\$ cd \char`~/repos/newawesomeproject.git\\
        \hspace{1cm}\$ git init -$\,$-bare
    }
}


\frame{
    \frametitle{Ah, I did something stupid\ldots}

    Recovery might be possible by looking into {\tt .git/logs/}.
}


\frame{
    \frametitle{Other commands}

    Graphs: {\tt git log -$\,$-graph}
    \bigskip

    More graphs: {\tt gitk -$\,$-all}
    \bigskip

    Tags: {\tt git tag}
    \bigskip

    Hooks: {\tt man githooks}; {\tt cd .git/hooks/}
    \bigskip

    Submodules: {\tt git submodule}
    \bigskip

    Rewrite history: {\tt  git filter-branch}
    \bigskip

    Collect garbage: {\tt git gc}
}



\end{document}


% vim: set sw=4 sts=4 et:
