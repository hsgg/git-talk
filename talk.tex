\documentclass{beamer}

\usepackage[utf8]{inputenc}
\usepackage[T1]{fontenc}

\usepackage{lmodern} % get rid of fontsize warning
\usepackage{tikz}
%\usetikzlibrary{calc,chains,shapes,positioning}
\usepackage{pgfplots}
\usepackage{verbatim}
\usepackage{amsmath}
\usepackage{textpos}
\usepackage{graphicx}
\usepackage[normalem]{ulem}
%\usepackage[colorlinks=false,pdfborder={0 0 0}]{hyperref}



\title{git - the stupid content tracker}
\subtitle{git eats trees. version control with git is fun.}
\author{Henry S. G. Gebhardt}
%\institute[The Pennsylvania State University]
%{Department of Astronomy and Astrophysics}
%\date{February 26, 2013}



\newcommand{\comp}[1]{{\tt #1}}


\begin{document}

\frame{\titlepage}

\section[Outline]{}
\frame{
    \frametitle{Outline}
    \tableofcontents
}


\section{Introduction to Version Control}
\frame{
    \frametitle{Version Control? Version Control.}

    \begin{itemize}
        \item Save history.
        \item See what changed.
        \item Share code.
        \item Merge code.
        \item Don't be a git.
        \item \ldots
    \end{itemize}

    Git, Mercurial, \sout{Bazaar}, \sout{SVN} (why bother?), \sout{CVS},
    \sout{Monotone}, \sout{DARCS}, \ldots

    ``Theory of Patches''
}


\section{Theory}
\frame{
    \frametitle{Git Theory}

    History is a DAG (directed acyclic graph). \emph{Explain graph.}
    \bigskip

    Distributed, not centralized. \emph{Every clone has the full history.}
    \bigskip

    There are \emph{plumbing} commands and \emph{porcelain} commands.
}


\section{git commands}
\frame{
    \frametitle{git cheat sheet}

    Here's the \emph{porcelain}:
    \bigskip

    \url{https://services.github.com/kit/downloads/github-git-cheat-sheet.pdf}
    \vfill
    Initialization once per machine:\\
    \hspace{1cm}Create the file {\tt \char`~/.gitconfig}.
}


\frame{
    \frametitle{Initial checkout}
    Existing repository:
    {\tt\\
        \hspace{0.3cm}\$ git clone ssh://git@git.psu.edu/hsg113/thesis.git
    }
    \vfill
    New repository:
    {\tt\\
        \hspace{1cm}\$ mkdir newrepo; cd newrepo\\
        \hspace{1cm}\$ git init
    }
}


\frame{
    \frametitle{git help command}

    Useful commands:

    \begin{description}[Other]
        \item[git status] Where am I?
        \item[git diff] What did I just do?
        \item[git diff -$\,$-staged] What will I do?
        \item[git log] What have I done?
        \item[gitk -$\,$-all] Let's climb trees!
        \item[git describe -$\,$-always -$\,$-tags] Who am I?
    \end{description}
}

\frame{
    \frametitle{blobs, trees, and commits are identified by their SHA1-sum}

    A hash is a (hopefully) unique number to identify some information, like a
    file.
    \medskip

    SHA1 is such a 256-bit number. It happens to be cryptographically secure.
    \medskip

    blobs, trees, and commits are identified by their SHA1 sum.
    \medskip

    \begin{itemize}
        \item[blob:] content in a file
        \item[treeish:] the directory structure, and the SHA1's of all blobs
        \item[commit:] the new tree SHA1's, the commit message, and all parent
            commits
    \end{itemize}

    $\Rightarrow$ efficient de-duplication and compression
}

\frame{
    \frametitle{Trees, yum!}
    \framesubtitle{git eats trees\ldots nom,nom}

    Branches are cheap!

    \begin{description}
        \item[git branch <name>] Let's make a new branch.
        \item[git branch -d] Never mind.
        \item[git checkout <name>] Let's climb over to that branch.
        \item[git checkout -b <newname> <starthere>] Checkout and make a new branch.
        \item[git merge <otherbranches>...] Trees eating trees!
        \item[git rebase -i <branchname>] I hate trees!
    \end{description}
}


\frame{
    \frametitle{Pushing and pulling}
    {\tt
        \hspace{1cm}\$ git push \\
        \hspace{1cm}\$ cd \char`~/repos/newawesomeproject.git\\
        \hspace{1cm}\$ git init -$\,$-bare
    }
}


\frame{
    \frametitle{Play with me!}
    \begin{center}
        {\tt \$ git svn}
        \vfill
        Works by calling ``{\tt git fast-import}''.
    \end{center}
}


\frame{
    \frametitle{Other commands}

    Create your own git server via ssh:
    {\tt\\
        \hspace{1cm}\$ mkdir -p \char`~/repos/newawesomeproject.git\\
        \hspace{1cm}\$ cd \char`~/repos/newawesomeproject.git\\
        \hspace{1cm}\$ git init -$\,$-bare
    }
    \vfill
    More fine-grained control: \url{http://gitolite.com/gitolite/}
    \vfill
    Hooks: {\tt man githooks}; {\tt cd .git/hooks/}
    \vfill
    Rewrite history: {\tt  git filter-branch}
}



\end{document}


% vim: set sw=4 sts=4 et:
